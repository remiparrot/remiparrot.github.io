\documentclass[11pt,a4paper,sans]{moderncv}

\moderncvtheme[blue]{casual} 
%\moderncvstyle{casual} 
%\moderncvcolor{blue}

\setlength{\hintscolumnwidth}{2cm} 

\usepackage[utf8]{inputenc}
\usepackage[scale=0.8]{geometry}
\usepackage{helvet}
\usepackage[french]{babel}
\usepackage{graphicx}
	\graphicspath{{../img}}

\makeatletter
\renewcommand*{\bibliographyitemlabel}{\@biblabel{\arabic{enumiv}}}
\makeatother

\name{Rémi}{Parrot}
\title{Maître de Conférence} 
%\address{1 rue de la Noë}{44300 Nantes}{France}
%\phone[mobile]{+33~6~45~73~27~67}
\email{remi.parrot@ec-nantes.fr}
\homepage{remiparrot.github.io}
%\social[linkedin]{pierre.durand}
%\social[twitter]{pierre.durand}
%\social[github]{pierre.durand}
%\extrainfo{informations complémentaires.}
\photo[64pt][0.4pt]{../img/photo2}
%\quote{Encore un titre}

\begin{document}

\maketitle

\section{Formation}
\cventry{2019--2022}{Diplôme de doctorat}{Centrale Nantes (ECN)}{Nantes}{\textit{Réseaux de Petri temporisés pour la synthèse de circuits pipelinés}}{}
\cventry{2015--2019}{Diplôme d'Ingénieur}{Centrale Nantes (ECN)}{Nantes}{\textit{Spécialité Informatique}}{}
\cventry{2013--2015}{Classes Préparatoire}{Lycée Bellevue}{Toulouse}{\textit{Spécialité PSI}}{}
\cventry{2013}{Baccalauréat S}{Lycée La Borde Basse}{Castres}{\textit{Spécialité SVT}}{}

\section{Expérience Professionnelle}
\cventry{Sep.~2024--\dots}{Maître de Conférence}{Centrale Nantes -- LS2N}{Automatique/Robotique -- STR}{Nantes--France}{}
\cventry{Sep.~2023--Aou.~2024}{ATER}{Centrale Nantes -- LS2N}{Automatique/Robotique -- STR}{Nantes--France}{Enseignement pour la formation ITII Systèmes Embarqués Communicants.}
\cventry{Nov.~2022--Aou.~2023}{Post-doc}{Uppsala Universitet}{CSD (Computer Science Division)}{Uppsala--Sweden}{Recherche sur le fuzzing de machine à états appliqué aux protocoles de communication.}
\cventry{Sep.~2019--Nov.~2022}{Thèse de Doctorat}{LS2N}{STR (Système Temps Réel)}{Nantes--France}{%
	\begin{itemize}
		\item Développement d'une approche de synthèse de pipeline avec des contraintes de temps et de ressources, basée sur des Réseaux de Petri Temporisés;
		\item Implémentation d'un outil de compilation Simulink vers VHDL;
		\item Construction d'un cours de VHDL pour des master M1 (master CORO à l'ECN);
		\item Enseignements et encadrement de projet d'étudiants (en master et dans le cursus ingénieur de l'ECN).
	\end{itemize}
	}
\cventry{Avr.--Aou. 2019}{Stage de 3ème année}{LS2N}{STR (Système Temps Réel)}{Nantes--France}{Travail de Recherche portant sur le control de modèles formels avec du temps et des coûts.}
\cventry{Nov.~2017--Avr.~2018}{Stage en année de Césure}{Valwin}{Service Informatique}{Nantes--France}{Amélioration des outils de producation de site de web de pharmacie.}
\cventry{Sep.--Oct. 2017}{Formation en année de Césure}{LS2N}{STR (Système Temps Réel)}{Nantes--France}{Portage de Trampoline RTOS sur un microcontrôleur SAM3X8E basé sur un processeur ARM Cortex-M3.}
\cventry{Avr.--Aou. 2017}{Stage 2ème année}{Universidad Complutense}{GASS (Grupo de Análisis, Seguridad y Sistemas)}{Madrid--Spain}{Travail de Recherche en analyse forensique.}
\cventry{Jui.--Aou. 2016}{Stage 1ère année}{CCL}{Service Informatique}{Castres--France}{Développement web et logiciel pour une entreprise commerciale.}

\newpage

\section{Projets de recherche}
\subsection{Post-doc}
\cvline{titre}{Fuzzing de machines à états pour les protocoles de communication}
\cvline{responsables}{Kostis Sagonas et Bengt Jonsson}
\cvline{description}{Le \textbf{fuzzing} est une technique de test qui consiste à fournir des entrées \textbf{aléatoires} au système testé jusqu'à ce qu'une erreur se produise.
Les protocoles de communication ont la spécificité d'implémenter (généralement) des \textbf{machines à états}.
Ces machines à états peuvent être apprises à l'aide de techniques d'\textbf{apprentissage de modèles}.
Puis, on peut guider le fuzzing afin d'explorer tous les états et donc le plus grand nombre de comportements possible du système.
}
\subsection{Thèse de Doctorat}
\cvline{titre}{Génération automatique de code VHDL pour des Chargeurs de Véhicules Électriques}
\cvline{direction}{Olivier H. Roux, Mikaël Briday et Malek Ghanes}
\cvline{description}{Cette thèse s'inscrit dans le cadre d'une collaboration avec l'entreprise automobile \textbf{Renault}, et a pour objectif de synthétiser des circuits sur \textbf{FPGA}, avec des contraintes de ressources et de temps.
Nous avons travaillé sur la \textbf{synthèse} d'un \textbf{pipeline optimal} et sur son utilisation pour le \textbf{multiplexage temporel}, c'est-à-dire la fusion de portions de circuits identiques en séquençant leur accès.
Pour résoudre ce problème, nous le réduisons à un \textbf{problème d'accessibilité optimale} dans un nouveau modèle de \textbf{réseau de Petri temporisé} que nous avons introduit, avec des transitions \emph{retardables} qui peuvent manquer leur date d'exécution et une action spécifique appelée \emph{reset} qui remet à zéro les horloges de toutes les transitions.
Nous avons étudié l'expressivité de ce modèle et proposé un algorithme d'exploration symbolique.}
%\cvline{date}{2019--2022}

\section{Langues}
\cvlanguage{Français}{Langue maternelle}{}
\cvlanguage{Anglais}{Lu, écrit, parlé}{(niveau C1)}
\cvlanguage{Espagnol}{Lu, écrit, parlé}{(niveau C1)}

\section{Outils}
\cvdoubleitem{Langages}{C, C++, Python, VHDL}{Compilation}{GCC,GDB,Xilinx Vivado}
\cvdoubleitem{Vérification}{Roméo, Uppaal}{Compilateur}{Flex, Bison, Galgas}
\cvdoubleitem{Versioning}{Git}{Formatage}{\LaTeX}

\section{Loisirs}
\cvline{ }{Escalade, Cirque, Bricolage, Arts Plastiques}

\newpage
\renewcommand{\refname}{Publications}
\nocite{*}
\bibliographystyle{plain}
\bibliography{../publications}

\end{document}
