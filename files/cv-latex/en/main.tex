\documentclass[11pt,a4paper,sans]{moderncv}

\moderncvtheme[blue]{casual} 
%\moderncvstyle{casual} 
%\moderncvcolor{blue}

\setlength{\hintscolumnwidth}{2cm} 

\usepackage[utf8]{inputenc}
\usepackage[scale=0.8]{geometry}
\usepackage{helvet}
\usepackage[french]{babel}
\usepackage{graphicx}
	\graphicspath{{../img}}

\makeatletter
\renewcommand*{\bibliographyitemlabel}{\@biblabel{\arabic{enumiv}}}
\makeatother

\name{Rémi}{Parrot}
\title{Teaching Assistant} 
\address{1 rue de la Noë}{44300 Nantes}{France}
%\phone[mobile]{+33~6~45~73~27~67}
\email{remi.parrot@ec-nantes.fr}
\homepage{remiparrot.github.io}
%\social[linkedin]{pierre.durand}
%\social[twitter]{pierre.durand}
%\social[github]{pierre.durand}
%\extrainfo{informations complémentaires.}
\photo[64pt][0.4pt]{../img/photo2}
%\quote{Encore un titre}

\begin{document}

\maketitle

\section{Education}
\cventry{2019--2022}{PhD}{Centrale Nantes (ECN)}{Nantes}{\textit{Timed Petri Nets for the synthesis of pipelined circuits}}{}
\cventry{2015--2019}{Graduate Engineering School}{Centrale Nantes (ECN)}{Nantes}{\textit{Computer Science and Research}}{}
\cventry{2013--2015}{Preparatory Classes}{Lycée Bellevue}{Toulouse}{\textit{Physics and Engineering}}{Intensive preparation for French Engineering School}
\cventry{2013}{High School Diploma}{Lycée La Borde Basse}{Castres}{\textit{Major in Science}}{}

\section{Experience}
\cventry{Sep.~2023--Aug.~2024}{Teaching Assistant}{École Centrale de Nantes}{Automatic/Robotics}{Nantes--France}{Teaching in training for Embedded System engineers.}
\cventry{Nov.~2022--Aug.~2023}{Post-doc}{Uppsala Universitet}{CSD (Computer Science Division)}{Uppsala--Sweden}{Research on stateful fuzzing for communication protocol.}
\cventry{Sep.~2019--Nov.~2022}{PhD Thesis}{LS2N}{STR (Système Temps Réel)}{Nantes--France}{%
	\begin{itemize}
		\item Research on the construction of a pipeline with time and resource constraints, using an approach based on timed Petri Nets;
		\item Implementation of a compilation tool from Simulink to VHDL;
		\item Creation of a VHDL course for master M1 (master CORO at ECN);
		\item Teaching and supervision of students projects at ECN (master and engineering students).
	\end{itemize}
	}
\cventry{Apr.--Aug. 2019}{3rd year Internship}{LS2N}{STR (Système Temps Réel)}{Nantes--France}{Research on the control of formal models with time and cost.}
\cventry{Nov.~2017--Apr.~2018}{Gap year Internship}{Valwin}{IT service}{Nantes--France}{Improvement of web site production tools for pharmacies.}
\cventry{Sep.--Oct. 2017}{Gap year formation}{LS2N}{STR (Système Temps Réel)}{Nantes--France}{Porting of Trampoline RTOS on microcontroller SAM3X8E based on processor ARM Cortex-M3.}
\cventry{Apr.--Aug. 2017}{2nd year Internship}{Universidad Complutense}{GASS (Grupo de Análisis, Seguridad y Sistemas)}{Madrid--Spain}{Research work of forensic analysis.}
\cventry{Jul.--Aug. 2016}{1st year Internship}{CCL}{IT service}{Castres--France}{Web and Software development for a commercial company.}

\newpage

\section{Research projects}
\subsection{Post-doc --- Uppsala Universitet}
\cvline{title}{Stateful fuzzing of communication protocols}
\cvline{PI}{Kostis Sagonas and Bengt Jonsson}
\cvline{description}{\textbf{Fuzzing} is a testing technique which consists in providing \textbf{random} inputs to the system under test until a bug occurs. Communication protocols have the specificity to (generally) implement \textbf{state machines}. Such state machines can be learned using \textbf{model learning} techniques. Finally, one can guide the fuzzing in order to explore all the states, and thus as many behaviour of the system as possible.}
\subsection{PhD Thesis --- École Centrale de Nantes, LS2N}
\cvline{title}{Timed Petri Nets for the synthesis of pipelined circuits}
\cvline{supervisors}{Olivier H. Roux, Mikaël Briday and Malek Ghanes}
\cvline{description}{This thesis was part of a collaboration with the automotive company \textbf{Renault}, with the objective of synthesizing resource and time constraint circuits on \textbf{FPGA}. 
We worked on the \textbf{synthesis} of \textbf{optimal pipeline} and on its usage for \textbf{time-multiplexing}, i.e., the merging of identical circuit portions by sequencing their access.
To solve this problem, we reduce it to an \textbf{optimal reachability problem} in a new \textbf{Timed Petri net} model that we introduced, with \emph{delayable} transitions that can miss their firing date and a specific action called \emph{reset} that resets the clocks of all transitions.
We studied the expressivity of this model and proposed a symbolic exploration algorithm.}
%This model turned out to be equivalent to a one-clock automaton.
%But an overclass, the Timed Petri Nets with delayable transitions (without reset), turns out to be incomparable, in terms of expressivity in weak semantics, with the classes of Temporal or Timed Petri nets in dense or discrete time.
%\cvline{date}{2019--2022}

\section{Languages}
\cvlanguage{French}{Native language}{}
\cvlanguage{English}{Fluent}{(level C1)}
\cvlanguage{Spanish}{Fluent}{(level C1)}

\section{Computing Tools}
\cvdoubleitem{Languages}{C, C++, Python, VHDL}{Compilation}{GCC, GDB, Xilinx Vivado}
\cvdoubleitem{Model Checking}{Roméo, Uppaal}{Compiler}{Flex, Bison, Galgas}
\cvdoubleitem{Versioning}{Git}{Formatting}{\LaTeX}

\section{Hobbies}
\cvline{ }{Climbing, handwork, juggling, art}

\newpage
\renewcommand{\refname}{Publications}
\nocite{*}
\bibliographystyle{plain}
\bibliography{../publications}

\end{document}
